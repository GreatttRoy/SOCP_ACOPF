\documentclass[12pt,a4paper]{article}

% 包引用
\usepackage[UTF8]{ctex}
\usepackage{amsmath}
\usepackage{amssymb}
\usepackage{amsthm}
\usepackage{graphicx}
\usepackage{booktabs}
\usepackage{longtable}
\usepackage{multirow}
\usepackage{array}
\usepackage{geometry}
\usepackage{hyperref}
\usepackage{xcolor}
\usepackage{listings}
%\usepackage{algorithm2e}
\usepackage{algorithmic}
\usepackage{algorithm}
\usepackage{enumitem}

% 页面设置
\geometry{left=2.5cm,right=2.5cm,top=2.5cm,bottom=2.5cm}

% 超链接设置
\hypersetup{
    colorlinks=true,
    linkcolor=blue,
    filecolor=magenta,
    urlcolor=cyan,
    citecolor=blue,
}

% 代码块设置
\lstset{
    basicstyle=\ttfamily\small,
    breaklines=true,
    frame=single,
    numbers=left,
    numberstyle=\tiny,
    keywordstyle=\color{blue},
    commentstyle=\color{gray},
    stringstyle=\color{red},
}

% 算法设置
\renewcommand{\algorithmicrequire}{\textbf{输入:}}
\renewcommand{\algorithmicensure}{\textbf{输出:}}
\renewcommand{\algorithmiccomment}[1]{\hfill$\triangleright$ \textit{#1}}

% 可跨页算法环境
\makeatletter
\newenvironment{breakablealgorithm}{%
  \begin{center}
    \refstepcounter{algorithm}%
    \hrule height.8pt depth0pt \kern2pt%
    \renewcommand{\caption}[2][\relax]{%
      {\raggedright\textbf{算法~\thealgorithm\quad}##2\par}%
      \kern2pt\hrule\kern2pt
    }%
}{%
    \kern2pt\hrule\relax
  \end{center}%
}
\makeatother

% 标题信息
\title{二阶锥松弛交流潮流优化的锥还原方法\\(SOCP-AC-OPF-Relaxation-Recovery-Cut-Adding-Method)}
\author{}
\date{}

\begin{document}

\maketitle

\tableofcontents
\newpage

% ==================== 符号表 ====================
\section{符号表}

\subsection{索引与集合}

\begin{longtable}{p{3cm}p{5cm}p{6cm}}
\toprule
符号 & 描述 & 说明 \\
\midrule
\endfirsthead
\toprule
符号 & 描述 & 说明 \\
\midrule
\endhead
$t$ & 时段索引 & $t \in \{0, 1, \ldots, T-1\}$ \\
$i, j, k$ & 节点索引 & $i, j, k \in N$ \\
$(i, j)$ & 支路索引 & 支路 $(i,j) \in E$ 表示连接节点 $i$ 和 $j$ 的支路 \\
$g$ & 发电机索引 & $g \in G$ \\
$dr$ & 可调负荷索引 & $dr \in DR$ \\
$N$ & 节点集合 & 配电网所有节点的集合,$|N|$ = 节点总数 \\
$E$ & 支路集合 & 配电网所有支路的集合,每条支路表示为 $(i, j)$ \\
$G$ & 发电机集合 & 所有发电机的集合,$|G|$ = 发电机总数 \\
$DR$ & 可调负荷集合 & 所有可调负荷的集合,$|DR|$ = 可调负荷总数 \\
$G_j$ & 节点 $j$ 的发电机集合 & 连接在节点 $j$ 的所有发电机 \\
$DR_j$ & 节点 $j$ 的可调负荷集合 & 连接在节点 $j$ 的所有可调负荷 \\
\bottomrule
\end{longtable}

\subsection{决策变量}

\subsubsection{支路变量}

\begin{longtable}{p{3cm}p{6cm}p{3cm}}
\toprule
符号 & 描述 & 单位 \\
\midrule
$P_{t,i,j}$ & 支路 $(i,j)$ 在时段 $t$ 的有功功率 & p.u. \\
$Q_{t,i,j}$ & 支路 $(i,j)$ 在时段 $t$ 的无功功率 & p.u. \\
$l_{t,i,j}$ & 支路 $(i,j)$ 在时段 $t$ 的电流幅值平方 $|I_{ij}|^2$ & p.u. \\
\bottomrule
\end{longtable}

\subsubsection{节点变量}

\begin{longtable}{p{3cm}p{6cm}p{3cm}}
\toprule
符号 & 描述 & 单位 \\
\midrule
$v_{t,i}$ & 节点 $i$ 在时段 $t$ 的电压幅值平方 $|V_i|^2$ & p.u. \\
$p_{t,i}$ & 节点 $i$ 在时段 $t$ 的有功注入功率 & p.u. \\
$q_{t,i}$ & 节点 $i$ 在时段 $t$ 的无功注入功率 & p.u. \\
\bottomrule
\end{longtable}

\subsubsection{发电机变量}

\begin{longtable}{p{3cm}p{6cm}p{3cm}}
\toprule
符号 & 描述 & 单位 \\
\midrule
$P_{g,t,g}$ & 发电机 $g$ 在时段 $t$ 的有功出力 & p.u. \\
$Q_{g,t,g}$ & 发电机 $g$ 在时段 $t$ 的无功出力 & p.u. \\
\bottomrule
\end{longtable}

\subsubsection{可调负荷变量}

\begin{longtable}{p{3cm}p{6cm}p{3cm}}
\toprule
符号 & 描述 & 单位 \\
\midrule
$P_{DR,t,dr}$ & 可调负荷 $dr$ 在时段 $t$ 的削减量(正数表示削减) & p.u. \\
%$P_{DR_j,t,dr}$ & 连接在节点 $j$ 的可调负荷 $dr$ 在时段 $t$ 的削减量(正数表示削减) & p.u. \\
\bottomrule
\end{longtable}

\subsection{网络参数}

\begin{longtable}{p{3cm}p{6cm}p{4cm}}
\toprule
符号 & 描述 & 单位 \\
\midrule
$r_{ij}$ & 支路 $(i,j)$ 的电阻 & p.u. \\
$x_{ij}$ & 支路 $(i,j)$ 的电抗 & p.u. \\
$V_{\min}$ & 节点电压幅值下限 & p.u.(默认 0.9) \\
$V_{\max}$ & 节点电压幅值上限 & p.u.(默认 1.1) \\
\bottomrule
\end{longtable}

\subsection{锥还原相关符号}

\begin{longtable}{p{3cm}p{7cm}p{3cm}}
\toprule
符号 & 描述 & 单位 \\
\midrule
$\Delta_{t,i,j}$ & 支路 $(i,j)$ 在时段 $t$ 的锥松弛间隙 & p.u.$^2$ \\
$\mathbf{x}_{ij}$ & 支路功率向量,$\mathbf{x}_{ij} = (P_{t,i,j}, Q_{t,i,j})^T$ & p.u. \\
$\mathbf{d}$ & 方向向量,$\mathbf{d} = (d_P, d_Q)^T$ & 无量纲 \\
$r$ & 收缩因子,$r \in [0, 1]$ & 无量纲 \\
$r_{\min}$ & 初始收缩因子 & 无量纲 \\
$r_k$ & 第 $k$ 层的收缩因子 & 无量纲 \\
$\mathcal{L}$ & 线性割约束集合 & - \\
$\mathcal{D}_\ell$ & 第 $\ell$ 层的方向集合 & - \\
$L_{\max}$ & 最大层数 & 正整数(默认 3-5) \\
$K_{\max}$ & 最大迭代次数 & 正整数(默认 50) \\
$\epsilon$ & 收敛容差 & p.u.$^2$(默认 $10^{-6}$) \\
\bottomrule
\end{longtable}

% ==================== 第1节:优化模型 ====================
\section{优化模型}

\begin{align}
\min_{P_{t,i,j}, Q_{t,i,j}, l_{t,i,j}, v_{t,i}, p_{t,i}, q_{t,i}, P_{g,t,g}, Q_{g,t,g}, P_{DR,t,dr}} \quad & \sum_{t=0}^{T-1} \sum_{dr \in DR_j} -P_{DR,t,dr} \\
\text{s.t.} \quad \nonumber \\
& v_{t,0} = 1.0, \quad \forall t \in \{0,1,\ldots,T-1\} \tag{1} \\
& p_{t,j} = \sum_{g \in G_j} P_{g,t,g} + \sum_{dr \in DR_j} P_{DR,t,dr} + P_{d,t,j}, \quad \forall t,j \in N \tag{2a} \\
& q_{t,j} = \sum_{g \in G_j} Q_{g,t,g} + Q_{d,t,j}, \quad \forall t, j \in N \tag{2b} \\
& p_{t,j} = \sum_{(j,k) \in E} P_{t,j,k} - \sum_{(i,j) \in E} \left( P_{t,i,j} - r_{ij} \cdot l_{t,i,j} \right), \quad \forall t,i,j,k \in N \tag{3a} \\
& q_{t,j} = \sum_{(j,k) \in E} Q_{t,j,k} - \sum_{(i,j) \in E} \left( Q_{t,i,j} - x_{ij} \cdot l_{t,i,j} \right), \quad \forall t,i,j,k \in N \tag{3b} \\
& v_{t,j} = v_{t,i} - 2\left( r_{ij} P_{t,i,j} + x_{ij} Q_{t,i,j} \right) + \left( r_{ij}^2 + x_{ij}^2 \right) l_{t,i,j}, \quad \forall t, (i,j) \in E \tag{4} \\
& \left(2P_{t,i,j}\right)^2 + \left(2Q_{t,i,j}\right)^2 + \left(l_{t,i,j} - v_{t,i}\right)^2 \leq \left(l_{t,i,j} + v_{t,i}\right)^2, \quad \forall t, (i,j) \in E \tag{5} \\
& P_{g,\min}(t) \leq P_{g,t,g} \leq P_{g,\max}(t), \quad \forall t, g \in G \tag{6} \\
& P_{DR,\min}(t) \leq P_{DR,t,dr} \leq P_{DR,\max}(t), \quad \forall t, dr \in DR \tag{7} \\
& V_{\min}^2 \leq v_{t,i} \leq V_{\max}^2, \quad \forall t, i \in N \tag{8} \\
& -2.5 \leq P_{t,i,j} \leq 2.5, \quad \forall t, (i,j) \in E \tag{9a} \\
& -2.5 \leq Q_{t,i,j} \leq 2.5, \quad \forall t, (i,j) \in E \tag{9b} \\
& 0 \leq l_{t,i,j} \leq 2.5, \quad \forall t, (i,j) \in E \tag{9c}
\end{align}

% ==================== 第3-6节省略,继续第7节 ====================
% 由于内容较长,这里仅展示关键部分结构

\newpage
\section{基于方向割的锥还原方法}

\subsection{问题背景}

在二阶锥交流潮流优化模型中,二阶锥松弛约束 (5) 对原始非凸约束进行了凸松弛:

\begin{equation}
l_{t,i,j} \cdot v_{t,i} = P_{t,i,j}^2 + Q_{t,i,j}^2 \quad \Rightarrow \quad \left\| \begin{bmatrix} 2P_{t,i,j} \\ 2Q_{t,i,j} \\ l_{t,i,j} - v_{t,i} \end{bmatrix} \right\|_2 \leq l_{t,i,j} + v_{t,i}
\end{equation}

\textbf{松弛的几何意义}:
\begin{itemize}
\item 原始约束要求点 $(P_{t,i,j}, Q_{t,i,j}, l_{t,i,j}, v_{t,i})$ 必须位于\textbf{二阶锥的表面}(锥面约束)
\item 松弛后的约束允许点位于\textbf{二阶锥的内部或表面}(锥约束)
\end{itemize}

在辐射状配电网络中,该松弛通常是紧的(tight),即最优解自然满足等式约束。但在某些情况下,松弛解可能落在锥的内部,此时需要进行\textbf{锥还原},将解投影到锥面上以获得原问题的可行解。

\subsection{锥松弛间隙检测}

设 SOCP 求解得到的最优解为 $(P^*_{t,i,j}, Q^*_{t,i,j}, l^*_{t,i,j}, v^*_{t,i})$,定义\textbf{锥松弛间隙}为:

\begin{equation}
\Delta_{t,i,j} = l^*_{t,i,j} \cdot v^*_{t,i} - \left[(P^*_{t,i,j})^2 + (Q^*_{t,i,j})^2\right]
\end{equation}

\textbf{间隙判定}:
\begin{itemize}
\item 若 $\Delta_{t,i,j} \approx 0$(在数值误差范围内),则松弛是紧的,无需还原
\item 若 $\Delta_{t,i,j} > \epsilon$(给定容差),则需要进行锥还原
\end{itemize}

\subsection{基于方向的锥还原方法}

当检测到松弛间隙时,采用基于方向的迭代方法将解还原到锥面上。该方法基于 \textbf{Cauchy-Schwarz 不等式},导出\textbf{方向割约束}。

\subsubsection{Cauchy-Schwarz 不等式}

对于支路 $(i,j)$,记 $\mathbf{x}_{ij} = (P_{t,i,j}, Q_{t,i,j})^T$,则二阶锥约束为:

\begin{equation}
\|\mathbf{x}_{ij}\|_2 \leq \sqrt{l_{t,i,j} \cdot v_{t,i}}
\end{equation}

对于任意方向向量 $\mathbf{d} = (d_P, d_Q)^T$,由 Cauchy-Schwarz 不等式可得:

\begin{equation}
\frac{\mathbf{d}^T \mathbf{x}_{ij}}{\|\mathbf{d}\|_2} \leq \|\mathbf{x}_{ij}\|_2 \leq \sqrt{l_{t,i,j} \cdot v_{t,i}}
\end{equation}

这表明:沿任意方向 $\mathbf{d}$ 的投影不超过 $\sqrt{l_{t,i,j} \cdot v_{t,i}}$。

\subsubsection{方向割约束}

为了收紧可行域,在方向 $\mathbf{d}$ 上添加\textbf{带收缩因子的方向割}:

\begin{equation}
\frac{\mathbf{d}^T \mathbf{x}_{ij}}{\|\mathbf{d}\|_2} \geq r \cdot \sqrt{l_{t,i,j} \cdot v_{t,i}}
\end{equation}

其中 $r \in [0, 1]$ 为收缩因子:
\begin{itemize}
\item $r = 1$:最紧的割,仅保留方向 $\mathbf{d}$ 上的锥面点
\item $r < 1$:较松的割,保留方向 $\mathbf{d}$ 附近一定角度范围内的锥面点
\end{itemize}

% ==================== 算法伪代码 ====================
\newpage
\subsection{算法总体流程}

层次化方向割锥还原算法由三个阶段组成:初始化、主迭代循环和终止处理。算法以 SOCP 松弛解作为起点,通过逐步添加方向割约束收紧可行域,驱动松弛解向锥面靠近,直至锥松弛间隙低于收敛容差 $\epsilon$ 或达到最大迭代次数 $K_{\max}$。

\paragraph{阶段一:初始化}
以初始 SOCP 解 $(P^*, Q^*, l^*, v^*)$ 为出发点,对每条支路 $(i,j)$ 提取当前功率方向 $\mathbf{d} = (P^*_{t,i,j}, Q^*_{t,i,j})^T$,计算初始收缩因子 $r_{\min} = \|\mathbf{d}\| / \sqrt{l^*_{t,i,j} v^*_{t,i}}$(即当前解在锥面方向上的投影比),并将对应的方向割约束加入约束集合 $\mathcal{L}$,同时置层计数 $\ell = 1$、迭代计数 $k = 0$、间隙历史集合 $\mathcal{H} = \emptyset$。

\paragraph{阶段二:主迭代循环}
每轮迭代包含以下步骤:

\textbf{步骤 1(SOCP 求解)}:在当前割约束集合 $\mathcal{L}$ 下求解 SOCP,得到新解 $(P^k, Q^k, l^k, v^k)$。

\textbf{步骤 2(间隙计算)}:对所有支路和时段计算锥松弛间隙 $\Delta^k_{t,i,j} = l^k_{t,i,j} v^k_{t,i} - [(P^k_{t,i,j})^2 + (Q^k_{t,i,j})^2]$,并取最大间隙 $\Delta^k_{\max}$ 记录至历史集合 $\mathcal{H}$。

\textbf{步骤 3(全局收敛检验)}:若 $\Delta^k_{\max} \leq \epsilon$,则所有支路的松弛均已收紧,返回当前解并标记为收敛。

\textbf{步骤 4(层内自适应方向更新)}:对仍有间隙($\Delta^k_{t,i,j} > \epsilon$)的支路,以当前解方向 $\hat{\mathbf{d}}$ 和当前层收缩因子 $r_\ell$ 生成新的方向割,补充进 $\mathcal{L}$。收缩因子 $r_\ell$ 在各层间线性递增(从 $r_{\min}$ 到 $1$),层号越高割越紧。

\textbf{步骤 5--6(停滞检测与层次提升)}:若近 $W_{\text{stag}}$ 次迭代的最大间隙改善量均低于 $\epsilon_{\text{stag}}$,判定当前层已停滞。此时若 $\ell < L_{\max}$,则提升至下一层:增大收缩因子 $r_\ell$,生成新层全局方向集 $\mathcal{D}_\ell$ 并添加对应的割约束,清空间隙历史重新监测;否则退出循环。

\paragraph{阶段三:终止处理}
若总迭代次数达到 $K_{\max}$ 仍未收敛,或层数达到最大层数 $L_{\max}$仍未收敛,返回当前最优解并标记为未完全收敛,供后续处理使用。
\subsection{算法伪代码}

\begin{breakablealgorithm}
\small
\caption{层次化方向割锥还原算法}
\label{alg:hierarchical_cone_recovery}
\begin{algorithmic}[1]
\REQUIRE SOCP 松弛解 $(P^*, Q^*, l^*, v^*)$,参数 $\epsilon, \epsilon_{\text{stag}}, W_{\text{stag}}, L_{\max}, K_{\max}$
\ENSURE 还原解 $(\bar{P}, \bar{Q}, \bar{l}, \bar{v})$,收敛状态
\STATE \textit{// 阶段一:初始化}
\STATE $k \leftarrow 0$,$\ell \leftarrow 1$
\STATE $\mathcal{L} \leftarrow \emptyset$ \COMMENT{方向割约束集合}
\STATE $\mathcal{H} \leftarrow \emptyset$ \COMMENT{间隙历史记录}
\STATE $\mathbf{d} \leftarrow (P^*_{t,i,j}, Q^*_{t,i,j})^T,\ \forall (i,j) \in E,\, t \in [0,T-1]$ \COMMENT{初始解的方向集合}
\STATE $\mathcal{L} \leftarrow \mathcal{L} \cup \left\{\mathbf{d}^T\mathbf{x}_{ij} / \|\mathbf{d}\| \geq r_1\sqrt{l_{t,i,j}v_{t,i}}\right\}$ \COMMENT{初始层的方向割约束}
\STATE $r_{\min} \leftarrow \|\mathbf{d}\| / \sqrt{l^*_{t,i,j}v^*_{t,i}}$ \COMMENT{初始收缩因子}
\STATE \textit{// 阶段二:主迭代循环}
\WHILE{$k < K_{\max}$}
    \STATE $k \leftarrow k + 1$
    \STATE \textit{// 步骤 1:求解带约束的 SOCP 问题}
    \STATE $(P^k, Q^k, l^k, v^k) \leftarrow \text{SOLVE\_SOCP}(\mathcal{L})$
    \STATE \textit{// 步骤 2:计算锥松弛间隙}
    \FORALL{$(i,j) \in E,\ t \in [0,T-1]$}
        \STATE $\Delta^k_{t,i,j} \leftarrow l^k_{t,i,j} \cdot v^k_{t,i} - \left[(P^k_{t,i,j})^2 + (Q^k_{t,i,j})^2\right]$
    \ENDFOR
    \STATE $\Delta^k_{\max} \leftarrow \max_{t,i,j} \Delta^k_{t,i,j}$;$\ \mathcal{H} \leftarrow \mathcal{H} \cup \{\Delta^k_{\max}\}$
    \STATE \textit{// 步骤 3:检查全局收敛}
    \IF{$\Delta^k_{\max} \leq \epsilon$}
        \RETURN $(P^k, Q^k, l^k, v^k)$,\textsc{Converged}
    \ENDIF
    \STATE \textit{// 步骤 4:层内自适应方向搜索(固定 $r_\ell$)}
    \STATE $r_\ell \leftarrow r_{\min} + (\ell-1)/(L_{\max}-1) \cdot (1 - r_{\min})$
    \FORALL{$(i,j) \in E,\ t \in [0,T-1]$ 且 $\Delta^k_{t,i,j} > \epsilon$}
        \STATE $\hat{\mathbf{d}} \leftarrow (P^k_{t,i,j}, Q^k_{t,i,j})^T / \left\|(P^k_{t,i,j}, Q^k_{t,i,j})\right\|$
        \STATE $\mathcal{L} \leftarrow \mathcal{L} \cup \left\{\hat{\mathbf{d}}^T\mathbf{x}_{ij} / \|\hat{\mathbf{d}}\| \geq r_\ell\sqrt{l_{t,i,j}v_{t,i}}\right\}$
    \ENDFOR
    \STATE \textit{// 步骤 5:判断层内停滞}
    \IF{$|\mathcal{H}| \geq W_{\text{stag}}$}
        \STATE $H_{\text{recent}} \leftarrow \text{LAST\_N\_ELEMENTS}(\mathcal{H}, W_{\text{stag}})$
        \STATE $\text{improvements} \leftarrow \{H_{\text{recent}}[i] - H_{\text{recent}}[i+1] \mid i = 1,\ldots,W_{\text{stag}}-1\}$
        \IF{$\max(\text{improvements}) < \epsilon_{\text{stag}}$}
            \STATE \textit{// 步骤 6:层次提升}
            \IF{$\ell < L_{\max}$}
                \STATE $\ell \leftarrow \ell + 1$;$\ r_\ell \leftarrow r_{\min} + (\ell-1)/(L_{\max}-1) \cdot (1 - r_{\min})$
                \STATE $\mathcal{D}_\ell \leftarrow \text{GENERATE\_GLOBAL\_DIRECTIONS}(\ell)$
                \FORALL{$(i,j) \in E,\ t \in [0,T-1],\ \mathbf{d} \in \mathcal{D}_\ell$}
                    \STATE $\mathcal{L} \leftarrow \mathcal{L} \cup \left\{\mathbf{d}^T\mathbf{x}_{ij} / \|\mathbf{d}\| \geq r_\ell\sqrt{l_{t,i,j}v_{t,i}}\right\}$
                \ENDFOR
                \STATE $\mathcal{H} \leftarrow \emptyset$ \COMMENT{清空历史,开始新层监测}
            \ELSE
                \STATE \textbf{break}
            \ENDIF
        \ENDIF
    \ENDIF
\ENDWHILE
\STATE \textit{// 阶段三:终止处理}
\RETURN $(P^k, Q^k, l^k, v^k)$,\textsc{NotFullyConverged}
\end{algorithmic}
\end{breakablealgorithm}

\subsection{算法复杂度分析}

\textbf{时间复杂度(单次迭代)}:

\begin{table}[h]
\centering
\begin{tabular}{lcc}
\toprule
操作 & 复杂度 & 说明 \\
\midrule
SOCP 求解 & $O(n^3)$ & $n$ 为决策变量数 \\
间隙计算 & $O(T \cdot |E|)$ & 遍历所有支路和时段 \\
方向割添加 & $O(T \cdot |E|)$ & 最多添加 $T \cdot |E|$ 个约束 \\
\textbf{总计} & \textbf{$O(n^3)$} & 由 SOCP 求解主导 \\
\bottomrule
\end{tabular}
\end{table}

\textbf{空间复杂度}:

\begin{table}[h]
\centering
\begin{tabular}{lcc}
\toprule
数据结构 & 复杂度 & 说明 \\
\midrule
决策变量 & $O(T \cdot |E|)$ & 支路变量和节点变量 \\
方向割约束集合 $\mathcal{L}$ & $O(K_{\max} \cdot T \cdot |E|)$ & 最坏情况 \\
\textbf{总计} & \textbf{$O(K_{\max} \cdot T \cdot |E|)$} & - \\
\bottomrule
\end{tabular}
\end{table}

\textbf{迭代次数}:
\begin{itemize}
\item 典型情况:10-30 次迭代
\item 最坏情况:$K_{\max} = 50$ 次迭代
\item 层数:通常 2-3 层
\end{itemize}


\end{document}
